% latex table generated in R 4.1.0 by xtable 1.8-4 package
% Fri Jul 23 16:10:22 2021
\begin{table}[p]
\caption[Description of vegetation types]{Climatic information and Dufr\^{e}ne-Legendre indicator species analysis for the vegetation type clusters identified by the PAM algorithm, based on basal area weighted species abundances. The three species per cluster with the highest indicator values are shown along with other key statistics for each cluster. MAP (Mean Annual Precipitation) and $\delta$T (Diurnal temperature range) are reported as the mean and 1 standard deviation in parentheses. Species richness is reported as the median and the interquartile range in parentheses.} 
\label{phen:clust_summ}
\begin{tabular}{S[table-format=1.0]S[table-format=3.0]ccccS[table-format=1.3]}
  \toprule
{Cluster} & {N sites} & {Richness} & {MAP} & {Diurnal dT} & {Species} & {Indicator value} \\
  \midrule
{\multirow{3}{*}{1}} & {\multirow{3}{*}{134}} & {\multirow{3}{*}{17(7)}} & {\multirow{3}{*}{1176(156.1)}} & {\multirow{3}{*}{13(1.5)}} & \textit{Brachystegia longifolia} & 0.397 \\ 
   &  &  &  &  & \textit{Uapaca kirkiana} & 0.390 \\ 
   &  &  &  &  & \textit{Marquesia macroura} & 0.285 \\ 
  \midrule
{\multirow{3}{*}{2}} & {\multirow{3}{*}{144}} & {\multirow{3}{*}{14(5)}} & {\multirow{3}{*}{955(172.8)}} & {\multirow{3}{*}{14(1.6)}} & \textit{Combretum molle} & 0.258 \\ 
   &  &  &  &  & \textit{Lannea discolor} & 0.228 \\ 
   &  &  &  &  & \textit{Combretum zeyheri} & 0.214 \\ 
  \midrule
{\multirow{3}{*}{3}} & {\multirow{3}{*}{243}} & {\multirow{3}{*}{17(6)}} & {\multirow{3}{*}{977(157.8)}} & {\multirow{3}{*}{14(1.5)}} & \textit{Julbernardia paniculata} & 0.559 \\ 
   &  &  &  &  & \textit{Brachystegia boehmii} & 0.540 \\ 
   &  &  &  &  & \textit{Pseudolachnostylis maprouneifolia} & 0.226 \\ 
  \midrule
{\multirow{3}{*}{4}} & {\multirow{3}{*}{96}} & {\multirow{3}{*}{14(6)}} & {\multirow{3}{*}{1012(185.5)}} & {\multirow{3}{*}{14(1.7)}} & \textit{Brachystegia spiciformis} & 0.582 \\ 
   &  &  &  &  & \textit{Cryptosepalum exfoliatum} & 0.285 \\ 
   &  &  &  &  & \textit{Guibourtia coleosperma} & 0.281 \\ 
   \bottomrule
\end{tabular}
\end{table}

