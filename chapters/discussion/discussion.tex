\begin{refsection}

\def\chaptertitle{Discussion}


\chapter{\chaptertitle}
\label{ch:discussion}

% Intro
This thesis aimed to improve understanding of the role of tree biodiversity in shaping the structure and of southern African woodlands. Dry tropical savanna-woodland mosaics form the dominant vegetation type in southern Africa \citep{Arino2010}. Their vegetation dynamics are complex \citep{Scholes1997}, their ecology is understudied compared to other dominant tropical vegetation formations \citep{Hill2010}, and they represent the largest uncertainty in models of the terrestrial carbon cycle \citep{Sitch2015, Williams2007}.

I framed the studies conducted in this thesis in relation to the ``Biodiversity-Ecosystem Function Relationship'' (BEFR), which predicts positive effects of biodiversity on ecosystem function, and is supported by hundreds of previous studies in other biomes \citep{Tilman2014, Plas2019}. BEFR theory predicts positive biodiversity effects mainly via the mechanism of niche complementarity \citep{Cardinale2009}, but it was unclear whether positive biodiversity effects would be observable in highly disturbed and resource-limited savanna-woodland ecosystems \citep{Tilman2014}. 

This thesis tests theory formulated in temperate and wet tropical ecosystems, in a highly disturbed and resource-limited system, southern African woodlands. The findings of this thesis contribute both to understanding the determinants of ecosystem function in southern African woodlands, and to BEFR theory more generally. To recap, the investigative chapters of this thesis explored: 

\begin{enumerate}
	\item{The effect of tree species diversity on above-ground woody biomass across southern African mesic savannas, and mediation of this effect by environment and vegetation composition (\autoref{ch:befr}).}
	\item{Land-surface phenology as a mechanism by which tree species diversity might increase gross primary productivity (\autoref{ch:phen}).}
	\item{The role of tree species diversity and composition in driving canopy complexity (\autoref{ch:tls}).}
	\item{Regional variation in species composition and woodland structure in miombo woodlands across southern Africa, with particular reference to the understudied Hu\'{i}la plateau, Angola (\autoref{ch:bicuar}).}
\end{enumerate}

These chapters aimed to advance understanding of three broader research questions:

\begin{enumerate}
\item{Is there a detectable relationship between biodiversity and ecosystem function across southern African woodlands, and to what extent is this mediated by environment and vegetation composition?}
\item{What are the possible mechanisms driving observed biodiversity-ecosystem function relationships in southern African woodlands?}
\item{How does the tree species diversity, composition and structure of mesic savannas vary across southern Africa?}
\end{enumerate}

Here, I synthesise the key findings of the thesis, set them in the context of the principal research questions, and discuss the implications of this thesis for understanding savanna ecology and biodiversity-ecosystem function theory. Furthermore, I provide a perspective for future research based on the results of this thesis.

\section{Summary of findings}
\label{discussion:sec:summ}
% Thesis overview
% What investigations did I conduct?
% Link back to original questions. Overall objectives
% Each chapter

\subsection{Evidence and implications of positive biodiversity effects in southern African woodlands}
\label{discussion:ssec:ecology}

In \autoref{ch:befr}, I conducted a regional study of the effects of tree species diversity, abiotic environment, and disturbance on above-ground woody biomass (AGB) and stand structure, using a network of plots distributed across southern Africa. I found that diverse plots generally held greater AGB, supporting the results of hundreds of previous studies of the biodiversity-ecosystem function relationship conducted in other natural ecosystems across the world \citep{Plas2019}, and of previous experimental studies \citep{Tilman2014}. This study however, joins only a handful of studies conducted in drylands and savannas \citep{Maestre2012, Grace2016, Plas2019, Clarke2017}. 

As well as the regional study presented in \autoref{ch:befr}, I conducted two more studies which demonstrated the influence of tree biodiversity on ecosystem function in southern African woodlands. In \autoref{ch:phen}, I found that species diversity correlated with increased growing season length and earlier pre-rain green-up. In \autoref{ch:tls} I found some evidence for a positive species diversity effect on woodland canopy complexity at hectare spatial scales. These studies are more mechanistic in nature, and contribute to understanding the functional contribution of different species and individuals in a way that broad studies such as that conducted in \autoref{ch:befr} cannot. 

In southern Africa, the human population is rapidly increasing and urbanising \citep{UN2018}, placing greater strain on natural resources in the region to provide ecosystem services, particularly provisioning services for timber, charcoal and fuel wood \citep{Wessels2013, Ryan2016}. Selective timber extraction in the region often targets large archetypal miombo species for their high quality timber \citep{Sitoe2010}. These species are predominantly from the Detarioideae subfamily, within the Fabaceae family. The results in this thesis suggest that removal of these individuals will have disproportionate negative effects on ecosystem function, functional resilience, and therefore on ecosystem service provision. 

The mutually supporting findings from \autoref{ch:befr}, \autoref{ch:phen}, and \autoref{ch:tls} suggest that biodiversity loss, particularly of large archetypal miombo tree species, negatively impacts ecosystem functions related to biomass and productivity, and the resilience of those functions. Southern African woodlands are a globally important carbon sink \citep{Grace2006, Pelletier2018}, and even small changes to the biomass or productivity could have global consequences \citep{Williams2005}. These findings highlight the potential for biodiversity loss to negatively impact the carbon sequestration potential of southern African woodlands, by reducing both productivity and potential woody biomass. Over the coming century, biodiversity `intactness' in southern Africa is expected to decrease by around 40\% \citep{Biggs2008}.

Previously, the influence of BEFR research on land management and policy has been limited \citep{Manning2019}. This thesis did not focus on strategies for management of natural resources, but a few tentative suggestions for management practices can be made. Variation in tree size was found to be a key vector for positive biodiversity effects, therefore appropriate management actions at small spatial scales should prioritise maintaining viable populations of species which differ in their growth strategy and average physiognomy. Furthermore, due to the dominant role of Detarioideae species on ecosystem functions related to productivity and biomass in miombo woodlands, I suggest that maintenance of these large canopy forming individuals should be prioritised.

Terrestrial Biosphere Models (TBMs) routinely use plant functional types (PFTs) to model spatial variation in biosphere function as it relates to carbon cycling \citep{Fisher2014}. PFTs represent broad groupings of species which co-exist to produce vegetation with unifying characteristics \citep{Bonan2002}. The studies conducted in this thesis also used vegetation type classifications to account for biogeographic variation in species composition that affects ecosystem function irrespective of species diversity. In these studies, it was consistently found that there was large variation in ecosystem function both within and among vegetation types. The vegetation types identified during the studies in this thesis were of greater specificity than the PFTs typically employed in TBMs \citep{Ustin2010, Krinner2005}, which raises the question of whether existing PFTs are detailed enough resolution to capture variation in ecosystem function and structural properties driven by biogeography and biodiversity \citep{Osborne2018, Torello2013}. As remote-sensing technology and its applications continue to develop rapidly, this thesis prompts discussion that variation in ecosystem properties detected via remote-sensing could be used to infer biodiversity and to refine maps of PFTs based on their functional signature. For example, in \autoref{ch:phen}, it was found that woodlands dominated by Combretaceae species have a greater peak in mid-rainy season productivity, but a shorter overall growing season than adjacent miombo woodlands. PFTs used in other studies however, routinely combine these woodlands into a single ``Deciduous woodlands'' PFT \citep{Krinner2005}. Variation in functional signature could be used to refine PFTs in a data-driven fashion, or potentially replace PFTs with continuous variation in traits and functional signature \citep{Peaucelle2019}. Of course, incorporating greater detail on vegetation type into TBMs would necessitate greater computing power and could result in greater uncertainty in model predictions if the information on vegetation types is incomplete or carries its own uncertainty. I suggest that earth system models up to regional spatial scales could look to incorporate these new developments which are supported by this thesis, but global models would have little to gain until higher resolution data are available that can surpass climatic correlates.

Most studies of biodiversity effects use biodiversity measures related to the number of species \citep{Tilman2014}. In this thesis, it was recognised that abundance evenness is another important facet of biodiversity that can provide further understanding of the mechanisms driving observed biodiversity effects \citep{Chalcraft2004}. In \autoref{ch:befr}, I found that species richness and abundance evenness had similarly strong factor loadings on the latent variable of species diversity, implying similar contributions to overall biodiversity effects. In \autoref{ch:phen}, I found that in miombo woodlands, while species richness caused an increase in cumulative EVI, longer season length, and earlier pre-rain green-up, abundance evenness caused a decrease in these metrics. This contrasting result suggests that selection effects may be driving much of the diversity effect in these woodlands, as an increase in evenness is necessarily associated with a decrease in the dominance of large canopy forming tree species which drive the phenological signal. Thus, while an increase in the number of species increases the likelihood of including these dominant species within the plot, an increase in evenness is associated with a reduction in their dominance. The dominant Detarioideae miombo species, grow to large canopy forming trees, with roots that can access deep groundwater reserves outside of the rainy season. Indeed, in the only non-miombo vegetation cluster encountered in data collection for \autoref{ch:phen}, abundance evenness did cause an increase in cumulative EVI and season length.

Alternatively, this result may say more about the effects of disturbance and succession on ecosystem function than they do about biodiversity effects \textit{per se}. As succession continues following a large disturbance, i.e. an intense fire, abundance evenness is expected to decrease, potentially approaching something similar to a Fisher's alpha species abundance curve, with a long tail of rare species \citep{Morozov2008, Sheil2001}. Greater abundance evenness isn't necessarily the `natural' or desirable state in wooded ecosystems. Thus, high abundance evenness is also possibly associated with tree size and other attributes related to succession following disturbance \citep{Holdo2006}, explaining the negative correlation of evenness with green-up lag, and cumulative EVI, as more mature woodland communities are expected to optimise productivity and resilience \citep{Hector2007}.

The analytical approach taken in \autoref{ch:befr} reflects the inter-connected nature of biodiversity, ecosystem function, and external factors such as disturbance and resource availability. Structural Equation Modelling (SEM) provides the means to effectively account for the simultaneous effect of environment on both species diversity and ecosystem function (\autoref{background:plas_befr}). Previous experimental studies have largely ignored the moderating effects of environment, and this has hampered efforts to scale up predictions of biodiversity effects to entire regions \citep{Thompson2021}. This novel analytical framework, coupled with the regional scale dataset, raises various points about the limitations and moderating factors of biodiversity effects in this biome.

% Across \autoref{ch:befr}, \autoref{ch:phen}, and \autoref{ch:tls}, water availability was repeatedly identified as a key underpinning factor which influenced tree species diversity, species composition, and ecosystem function. The results from \autoref{ch:befr} however, suggest that even when climatic variation is accounted for in statistical modelling, positive biodiversity effects are still detectable, and differ among vegetation types. 

\subsection{Thresholds on biodiversity effects}
\label{discussion:ssec:thresholds}

In the \hyperref[ch:intro]{introduction} and \hyperref[ch:background]{background} to the thesis, I suggested that there may be key climatic and structural thresholds below which biodiversity effects driven by competition do not occur in dry tropical woodlands. This hypothesis is drawn from fundamental biodiversity-ecosystem function theory, which suggests that niche complementarity effects are predicated on inter-specific competition \citep{Isbell2013}. In dry tropical woodlands however, resource limitation, environmental stress, and disturbance may limit the role of competition, as abundance is kept below carrying capacity \citep{Sankaran2005}. In \autoref{ch:befr}, I found that biodiversity effects became stronger with increasing organismal density, i.e. tree stem density. Tree stem density was in turn driven by water availability and reduced by fire frequency. This finding suggests that there are indeed environmental factors which either limit or mask the role of biodiversity in driving ecosystem function in stressful conditions. The majority of previous biodiversity-ecosystem function relationship studies do not incorporate the effects of environment, and instead focus on a given location where environmental conditions are assumed to vary little across the study site \citep{Cardinale2009, Plas2019}. The threshold effects identified in \autoref{ch:befr} could explain the varying strength and sign of the observed BEFR in other studies \citep{Liang2016}.

A number of studies in southern African woodlands have investigated ecological thresholds that determine the transition from open savanna to a closed canopy forest-like condition \citep{Staver2011, Hirota2011, Staver2017}. These studies have identified disturbance thresholds that maintain ``alternative stable states'' under similar climatic conditions. The findings in \autoref{ch:befr} and \autoref{ch:tls}, suggest that tree species biodiversity itself could moderate these threshold conditions, by allowing greater biomass, greater diversity in tree physiognomy, greater canopy closure, and taller canopies, in more diverse woodlands. This variation in canopy structure with diversity is expected to further moderate disturbance through its effect on understorey grass fuel load.

In \autoref{ch:phen}, it was found that in Zambian woodlands, a stronger positive evenness effect was observed on cumulative EVI and growing season length in vegetation where there was more moisture limitation. In higher precipitation woodlands, it appears that the dominant archetypal miombo tree species can grow to large canopy forming trees, and these individuals define cumulative EVI as a result. Meanwhile, in drier woodlands, a genuine niche complementarity effect occurs. In these drier woodlands, higher species diversity provides ecosystem level resilience to drought by increasing the breadth of water use strategies. This aligns with \citet{Ratcliffe2017}, who conducted a study in dry Mediterranean woodlands, finding that facilitation and niche complementarity effects increased under greater seasonal drought stress.

\subsection{Mechanisms for positive biodiversity effects in mesic savannas}
\label{discussion:ssec:mechanism}

A multitude of previous studies have found that biodiversity positively correlates with productivity, AGB \citep{Liang2016}, and the resilience of these ecosystem functions to environmental perturbations \citep{Mori2012}, but fewer have investigated the ecological mechanisms by which biodiversity affects ecosystem function \citep{Barry2019}. Identifying these pathways of effect not only improves understanding of the ecological processes underlying these observed effects, but also builds a more nuanced understanding of how ecosystem structure and function may change as a result of climate change or changes to disturbance regime \citep{Huston2014}.

\subsubsection{Land-surface phenology}
\label{discussion:sssec:phen}

In \autoref{ch:phen}, I found that species diversity led to increased growing season length, and earlier pre-rain green-up. Land-surface phenology plays and important regulating role in the global carbon, water, and nitrogen cycles \citep{Richardson2013}, and is frequently incorporated into Terrestrial Biosphere Models (TBMs), as a proxy for gross primary productivity \citep{Bloom2016}. In southern African woodlands particularly, a large body of scholarship has grown to investigate the pre-rain green-up phenomenon \citep{Ryan2017, Adole2018a}. My findings suggest that variation in phenological strategy among tree species is one mechanism by which species diversity increases resilience to drought and maximises productivity in water-limited ecosystems \citep{Stan2019, Morellato2016}. In southern African woodlands especially, where woody growth is highly seasonal and ecosystem structure is determined by disturbance, controls on land-surface phenology constitute a key ecosystem function. Patterns of growth determine which individuals are able to escape the ``fire-trap'' \citep{Dantas2013}, with feedbacks between fire and woody growth determining whether a patch remains as open grassy savanna or woodland \citep{Staver2011}.

While pre-rain green-up has been explained by a multitude of climatic and now biodiversity variables, controls on senescence have not received as much attention \citep{Gallinat2015}. Previous studies have suggested that senescence is largely driven by date of green-up, suggesting that resource limitation limits the length of the growing season \citep{Zani2020}, while others have suggested the end of season signal is dominated by grasses which more closely follow rainfall patterns. In \autoref{ch:phen}, I found that the proportion of larger trees in plots also caused delayed senescence, extending the growing season and ultimately increasing productivity. This is a novel result which demonstrates the effect of large trees and older woodland patches in providing a buffer to environmental factors, i.e. the decline of water availability towards the end of the growing season. This result is in contrast to previous studies in forests, which found that larger trees are at greater risk from drought due to an increased risk of embolism \citep{Bennett2015}. I suggest that in drought-adapted savannas with a distinct seasonal rainfall pattern, larger trees can access deep groundwater reserves, making them more resilient to oscillations in rainfall. In deciduous water-limited savannas, large trees could also provide resilience to longer-term drought, which is expected to become more prevalent as human-induced climate change progresses \citep{Kusangaya2014}. 

The incorporation of biodiversity and biotic change into carbon cycle modelling has been limited \citep{Ahlstrom2015, Bodegom2011}. Large uncertainties in the effects of diversity on Gross Primary Productivity, and difficulties in measuring diversity over regional spatial scales has hampered attempts to include this data in large spatial scale models. The results presented in \autoref{ch:phen} provide a link by demonstrating a strong positive relationship between species diversity and cumulative EVI, which itself correlates with gross primary productivity (GPP) \citep{Sjostrom2011}.
% Earlier?

\subsubsection{Canopy complexity}
\label{discussion:sssec:canopy}

Previous work has linked canopy complexity to various ecosystem functions, such as increased productivity \citep{Gough2019, Hardiman2011}, increased resilience of productivity \citep{Pretzsch2014}, and recently a mechanistic link has been found between canopy complexity and solar-induced chlorophyll fluorescence triggered by photosynthesis \citep{Regaieg2021}. At local spatial scales, canopy complexity is highly variable, with much variation not accounted for in existing studies \citep{Guan2014}. In \autoref{ch:tls}, I hypothesised that species diversity might explain some of this local scale variation. I found that tree species diversity in miombo woodlands correlated with increased canopy closure, taller canopies, and increased foliage density. Species diversity effects were weak, though this was not unexpected, as stochasticity in canopy structure caused by small spatial scale environmental heterogeneity, and landscape history, could have masked biodiversity effects at this spatial scale. Indeed, at plot scales, biodiversity effects were stronger.

Unlike in \autoref{ch:befr}, where tree size diversity was identified as a mechanism by which species diversity caused increases in woody biomass, in \autoref{ch:tls} tree size variation had negligible effects. Tree size variation only affected subplot canopy layer diversity. However, species diversity did still influence both tree size variation and canopy complexity separately. This suggests the presence of species-specific differences in the plasticity and physiological limits of crown shape that are independent of tree size. It also points to a genuine species diversity effect acting through structural diversity in these woodlands.
% Why are ch:befr and ch:tls different in this regard?

Species diversity was found to increase canopy complexity metrics related to foliage volume, i.e. canopy closure and foliage density, but decreased metrics related to spatial heterogeneity of foliage, i.e. canopy surface roughness and whole-canopy rugosity. I suggest that in the sparser canopies encountered in the sampled woodlands, variation in spatial heterogeneity was driven by empty space within the canopy, i.e `between-canopy gaps', whereas in forests spatial heterogeneity is related more to `within-canopy gaps' and overall foliage density. Therefore, as species diversity increased in savanna-woodlands, canopy density also increased, reducing the proportion of empty space in the canopy, thus reducing these measures of heterogeneity. This finding therefore prompts discussion of the suitability of these metrics for estimating canopy complexity in savannas, and their ecological significance. \citet{Hardiman2011} assert that canopy rugosity, i.e. whole canopy heterogeneity in foliage density, drives Net Primary Productivity (NPP) by increasing light transmission through the canopy profile, and this is supported by earlier conceptual work \citep{Horn1971}. However, this previous work was conducted in forests, with a denser and more closed canopy. In savannas, where light penetration is less of a limiting factor to tree growth \citep{Frost1996}, foliage heterogeneity may not exert such an effect on productivity. 

\subsubsection{Structural diversity}
\label{discussion:sssec:struc}

% Does this repeat an earlier paragraph?
Large trees drive phenology (\autoref{ch:phen}), and hold disproportionate biomass (\autoref{ch:bicuar}). Large sized trees are frequently identified in other studies as contributing disproportionately to ecosystem function, and resilience of function \citep{Ali2021}. In \autoref{ch:befr}, I found that structural diversity, i.e. variance in stem diameter and tree height, led to greater AGB, and provided an indirect pathway for the effect of species diversity on AGB \citep{Ali2016, Pedro2017}. In \autoref{ch:tls}, I found no evidence that structural diversity influenced canopy complexity. I did however, find that species diversity influenced both tree size diversity and canopy complexity separately, suggesting a partial decoupling of tree size and foliage volume. This result highlights variation among species in their average physiognomy that is independent of their size, and reinforces the conclusion that structural diversity is a key mechanism in southern African woodlands by which species diversity drives ecosystem function. This result supports seminal work describing southern African woodlands which described them as harbouring great variety of tree functional forms with low functional redundancy, despite their low species diversity compared to other wooded ecosystems \citep{Solbrig1996}. Taken further, this finding could imply that biodiversity loss in these woodlands could have greater negative effects on ecosystem function than in more diverse ecosystems, due to their low functional redundancy.

In \autoref{ch:tls}, I considered not only the diversity of species, but also their spatial arrangement within the plot. It was hypothesised that greater spatial mingling of species would lead to greater canopy closure, under the assumption that a diversity of functional forms in the local neighbourhood would reduce competition and increase foliage density. In practice however, the opposite was found, where increased spatial mingling was associated with lower canopy closure and shorter canopy height. I interpreted that this result occurred due to the covariation of both canopy closure and spatial mingling with disturbance. Disturbance by fire is expected to promote conspecific spatial clustering, and simultaneously reduce canopy closure \citep{Martens2000}. This result highlights the complexity of savanna ecosystem processes compared to forests in temperate and wet tropical regions, and the inter-dependence of diversity, environment, and disturbance, with feedback effects which can lead to non-intuitive outcomes in vegetation dynamics. Theory developed in competition dominated wooded ecosystems cannot be naively applied to savannas without considering the role of disturbance.

\subsection{Regional variation in woodland structure and tree diversity across the miombo ecoregion}
\label{discussion:ssec:veg_type}

Biodiversity is defined most frequently in studies of the biodiversity-ecosystem function relationship in terms of the variety of organisms which co-inhabit a given ecosystem, i.e. alpha diversity. Variation in species composition over space, i.e. beta diversity, constitutes another perspective on biodiversity that was investigated in this thesis. All four investigative chapters presented in this thesis included a `vegetation type' element in their statistical analyses in order to describe variation in ecosystem function among plots that was not due to variation in the diversity of organisms found within plots, but due to the particular identity of the species combinations found therein. Distinct vegetation formations are easily distinguishable in southern African woodlands, with visible differences in structure and function driven by functional differences between their dominant tree species \citep{Solbrig1996}. Distinct vegetation types arise as a result of climatic variation, disturbance history, and biogeography \citep{Fayolle2018}. I found that southern African woodland vegetation types differed in their aboveground biomass (\autoref{ch:befr}), growing season phenological patterns (\autoref{ch:phen}), and markedly in their canopy structure (\autoref{ch:tls}). They also differed in the strength of observed biodiversity effects. 

In general, miombo woodlands tended to have stronger biodiversity effects than mopane, ex-Acacia and \textit{Baikiaea} woodland types. Miombo woodlands were found to contain more tree species than other vegetation types, and generally existed in less environmentally stressful areas. \citet{Mensah2020} found that woodlands and forests in West Africa had positive tree species richness effects, while in sparse savannas richness effects were negligible. Miombo woodlands are frequently misclassified as forest by vegetation maps and remote-sensing studies, as they often possess a contiguous but sparse canopy \citep{Solbrig1996}. Possibly the tree communities in the miombo woodlands encountered in this thesis are structured more similar to forests than to arid savannas with a non-contiguous canopy. While resource limitation does appear to strongly moderate biodiversity effects, disturbance by fire, which is common across miombo woodlands and other savanna types in southern Africa \citep{Saito2014}, does not appear to have the same weakening effects on the biodiversity-ecosystem function relationship. Possibly this is because miombo tree species are highly adapted to deal with fire \citep{Dantas2013}. 

The plots used in the thesis include the major savanna-woodland vegetation types in southern Africa \citep{White1983}, notably multiple forms of miombo, \textit{Baikiaea} woodlands, mopane woodlands, and ex-Acacia woodlands. Arid savannas, which are found at the southern and north-eastern limits of the miombo ecoregion, were not considered in the thesis. In \autoref{ch:befr} it was demonstrated that water availability remained a key driving force behind the detection of biodiversity effects. Similarly in \autoref{ch:tls} the drier ex-Acacia savannas did not show positive biodiversity effects on canopy complexity. I suggest that in severely water-limited savannas, biodiversity effects driven by niche complementarity may be negligible. However, facilitation effects may be more important in these systems than could be detected in this thesis. \citet{Ratcliffe2017}, working in dry Mediterranean forests, found that the strength of the effect of tree species richness on many ecosystem functions increased as water availability decreased. 

In \autoref{ch:tls}, I discussed how ex-Acacia species develop sparser canopies, with shorter stature and wider crowns. This condition reduces canopy complexity and prevents the formation of a vertically stratified complex canopy. This means that acacia savannas may be less likely to close their canopy than miombo woodlands, for instance. This could be conflated with environment however, as Acacia savannas occur most often in water-limited and heavily grazed habitats, where complex canopies are less likely to form anyway \citep{Archibald2003}.

In \autoref{ch:bicuar}, I found that woodlands at the western edge of the miombo are floristically distinct from those in the east. Core `miombo' species are generally shared by sites across the region, while the floristic variation arises mainly from smaller understorey species that inhabit disturbed areas. In \autoref{ch:tls} and \autoref{ch:bicuar}, I showed how even with similar species composition, woodland structure differs among miombo woodlands in different parts of the miombo ecoregion. Woodlands in Tanzania had taller canopies and wider spreading crowns than those in Angola. These differences are likely due to climate. Woodlands encountered in Tanzania were warmer and wetter than those in southwest Angola. Miombo woodlands studied in southwest Angola are possibly closer to the edge of their bioclimatic envelope, explaining the shorter and less full canopies \citep{Scholes2002}. These woodlands may therefore be at greater risk from the effects of climate change, which is predicted to cause reductions in precipitation across southern Africa over the coming century \citep{Kusangaya2014}.

\section{Limitations and directions for future research}
\label{discussion:sec:future}

While hundreds of experimental studies of the biodiversity-ecosystem function relationship have been conducted, most of these have used mesocosms or grass patches as the unit of study. Few have experimentally manipulated the diversity of organisms as large as trees \citep{Huang2018}, and none have done this in dry tropical woodlands or savannas. The current trend is to conduct BEFR studies in real-world ecosystems, in order to test theory established in experimental settings \citep{Plas2019}. One key limitation of these real-world studies, and of the studies conducted in this thesis, is that they are ultimately correlative. While variation in environment, species composition, and disturbance were accounted for in statistical analyses, there is still the potential for third variable effects and even reverse causation, where ecosystem function may actually influence biodiversity \citep{Eisenhauer2016}. With the proliferation of plot-based studies in southern African woodlands \citep{Ryan2020}, a valuable exercise would be scaling up previous experiments to whole woodland plots, and manipulating their species diversity over the course of multiple growing seasons to track changes in ecosystem functions related to ecosystem productivity, canopy complexity, and phenology. Such an experimental study would add further clarification and potentially support the findings of this thesis.

This thesis focussed on the biodiversity of trees, and on ecosystem functions related to trees. Southern African woodlands however, are characterised by the coexistence of trees and herbaceous understorey vegetation, notably C\textsubscript{4} grasses. Grasses in savannas have been estimated to account for between 40\% \citep{Whitley2011} and 59\% \citep{Lloyd2008} of total GPP, and provide other ecosystem functions not performed by other functional groups \citep{Soliveres2016}. While positive biodiversity effects were observed for ecosystem functions related to tree cover, productivity, and woody biomass, other important ecosystem functions not related to trees were ignored, such as grass biomass, soil nutrient retention, and wildlife fodder provision. In \autoref{ch:tls}, I concluded that in miombo woodlands, greater tree species diversity could accelerate woody thickening and potentially woody encroachment, which would increase values of various ecosystem functions related to tree growth. The vast majority of studies of the biodiversity-ecosystem function relationship in wooded ecosystems focus on ecosystem functions related to primary productivity, biomass, and carbon fixation, presumably because these are interpreted as globally important \citep{Grace2004}, but possibly also because they are straightforward to quantify. Previous studies have suggested that when multiple functions are considered, that the positive effect of biodiversity increases \citep{Hector2007}, but few studies have considered combinations of ecosystem functions generated by different functional groups of species, i.e. trees and grasses \citep{Hooper2005}. While tree species diversity appears to promote ecosystem functions related to tree cover and woody biomass in southern African woodlands, it could decrease ecosystem functions dominated by the non-tree portion of the biota, such as soil water retention and soil carbon retention \citep{Oliveira2005}.
% Future studies should

As well as altering ecosystem function, by reducing grass growth and understorey plant production, greater woody biomass and tree cover could reduce overall biodiversity by excluding herbaceous species and the many animal species that rely upon them \citep{Ratajczak2012, Grellier2013}. It is a frequent adage among savanna scientists that the majority of biodiversity in savannas lies in the herbaceous layer \citep{Veldman2015}. If the goal is to maintain and improve biodiversity in savannas, irrespective of functional group or trophic level, perhaps tree cover should be kept at an intermediate level to prevent exclusion of herbaceous species. If the goal is to sequester carbon however, perhaps tree biodiversity should be a greater concern.

Now that a broad corpus of literature exists which demonstrates positive biodiversity effects in a variety of ecological contexts, I suggest that future studies work to more fully understand the functional contribution of different species, and optimal combinations of species which maximise multiple facets of ecosystem function. \autoref{ch:tls} of this thesis was particularly enlightening as I was able to investigate the effects of specific species combinations on canopy complexity. I concluded that species from the Detarioideae subfamily, i.e. the dominant miombo canopy tree species, had a greater effect on canopy complexity than other species. Similarly, in \autoref{ch:phen}, I found that Detarioideae miombo species dominated the phenological signal, particularly during the green-up and senescence phases. Variation in growth strategy and trait values among species affects their contribution to ecosystem function, aligning somewhat with Grime's Mass Ratio Hypothesis \citep{Grime1998}, which suggests that it is not the breadth of filled niche space that determines ecosystem function, but the ability of the most abundant species to optimise ecosystem function. Indeed, there is no logical reason why the species should contribute equally to a given ecosystem function, and it is sensible to predict that species maximise different ecosystem functions, depending on their life history strategy \citep{Bengtsson1998}. This is not to diminish the role of species richness in driving ecosystem function, as niche complementarity may still operate under this framework.

% There is a temptation to scale up work to larger spatial scales, to produce maps and global models that address grand ecological questions. However, if the goal is to make recommendations to individual land managers with the hope of conserving southern African woodlands, small scale studies and understanding neighbourhood effects such as those studied in \autoref{ch:tls} are potentially more useful.

Even with the 1 ha plot size used in \autoref{ch:tls}, there was still a great deal of stochasticity in canopy complexity and stand structure among plots, which made it difficult to unequivocally attribute diversity effects to canopy complexity metrics. In \autoref{ch:bicuar}, similarly there was a lot of stochastic variation in stem density within plots, which led to wide variation in reported AGB. Spatial clustering of trees and the self-reinforcing effects of disturbance by fire and herbivory in mesic savannas produces a patchwork mosaic of open grassy patches and woodland patches \citep{Staver2011, Schertzer2015}. Plot-based studies can fail to capture this spatial variation in ecosystem structure and result in biased estimates of landscape-scale ecosystem properties such as AGB. I suggest that future studies in the region move towards fewer, larger plots with a concentration of measurement types in one area, rather than many smaller and less provisioned plots \citep{Kreyling2018}. The emergence of supersites in other regions of the world shows that this method of data collection is becoming more popular \citep{Schepaschenko2019}, but as yet there are no sites in the miombo eco-region which would qualify for supersite status by the major supersite networks. In \autoref{ch:legacy} I proposed that the plots set up as part of this thesis would be suitable candidates for supersite status, and I hope that in the coming years the profile of this understudied region of southern Africa is elevated to match that of the longer-running plot sites in Mozambique \citep{Ryan2011} and Tanzania \citep{McNicol2018a}.

\section{Concluding remarks}
\label{discussion:sec:conclusion}

In this thesis I demonstrated the role of tree species diversity as a determinant of ecosystem structure and function in southern African woodlands. This thesis has provided nuance on the vegetation dynamics of this understudied biome. It has also contributed valuable case studies to the already broad corpus of biodiversity-ecosystem function research, testing theory developed in temperate and wet tropical systems in a highly disturbed and resource-limited system. In doing so, I have demonstrated the limits of biodiversity effects, and raised questions about the generality of the frequently reported biodiversity-ecosystem function relationship. I suggest that future work continues to focus on ecological mechanisms driving observed biodiversity effects, and broadens the scope of studies to include multiple functional groups of organisms along with environmental effects, to account for their inter-connected nature. I hope that future studies can leverage the current explosion in plot-based ecological monitoring data and novel data collection techniques employing state of the art technology to delve deeper into this pertinent field of study, which sits at the heart of community ecology, conservation, and earth system modelling.

\newpage{}
\begingroup
\setstretch{1.0}
\printbibliography[heading=subbibintoc]
\endgroup

\end{refsection}

