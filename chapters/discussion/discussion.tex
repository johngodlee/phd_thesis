\begin{refsection}

\def\chaptertitle{Discussion}


\chapter{\chaptertitle}
\label{ch:discussion}

% Intro
This thesis aimed to improve understanding of the role of tree biodiversity in shaping the structure and function of southern African woodlands. Dry tropical savanna-woodland mosaics form the dominant vegetation type in southern Africa \citep{}, though they represent the largest uncertainty in models of the terrestrial carbon cycle \citep{Ahlstrom2015}. Their vegetation dynamics are complex \citep{}, and their ecology is generally understudied compared to other dominant tropical vegetation formations \citep{}. I contextualised the studies conducted in this thesis against the ``biodiversity-ecosystem function relationship'' (BEFR), which predicts positive effects of biodiversity on ecosystem function, and which is supported by hundreds of previous studies in other biomes \citep{}. BEFR theory predicts positive biodiversity effects mainly via the mechanism of niche complementarity \citep{}, but it was unclear whether positive biodiversity effects would be observable in highly disturbed and resource-limited savanna-woodlands ecosystems \citep{}. The findings of this thesis contribute both to understanding of the determinants of ecosystem function in southern African woodlands, and to BEFR theory more generally, by testing theory formulated in temperate and wet tropical ecosystems \citep{}, in a highly disturbed and resource-limited system.

To recap, the investigative chapters of this thesis explored: 

\begin{enumerate}
	\item{The effect of tree species diversity on above-ground woody biomass across southern African mesic savannas, and mediation of this effect by environment and vegetation composition (\autoref{ch:befr}).}
	\item{Land-surface phenology as a mechanism by which tree species diversity could increase gross primary productivity (\autoref{ch:phen}).}
	\item{The role of tree species diversity and composition in driving canopy complexity (\autoref{ch:tls}).}
	\item{Regional variation in species composition and woodland structure in miombo woodlands across southern Africa, with particular reference to the understudied Hu\'{i}la plateau, Angola (\autoref{ch:bicuar}).}
\end{enumerate}

These chapters aimed to advance understanding of three broader research questions:

\begin{enumerate}
\item{Is there a detectable relationship between biodiversity and ecosystem function across southern African woodlands, and to what extent is this mediated by environment and vegetation composition?}
\item{What are the possible mechanisms driving observed biodiversity-ecosystem function relationships in southern African woodlands?}
\item{How does the tree species diversity, composition and structure of mesic savannas vary across southern Africa?}
\end{enumerate}

Here, I synthesise the key findings of the thesis, set them in the context of the principal research questions, and discuss the implications of this thesis for understanding savanna ecology and biodiversity-ecosystem function research. Furthermore, I provide a perspective for future research based on the results in this thesis.

% Summary of main findings
\section{Summary of findings}
% Thesis overview
% What investigations did I conduct?
% Link back to original questions. Overall objectives
% Each chapter

\subsection{Evidence of positive biodiversity effects in southern African woodlands}

In \autoref{ch:befr}, I conducted a regional study of the effects of tree species diversity, environment, and disturbance on above-ground woody biomass (AGB) and stand structure, using a distributed network of plots from the SEOSAW network. I found that diverse plots generally held greater AGB, supporting the results of hundreds of previous studies of the biodiversity-ecosystem function relationship conducted in other natural ecosystems across the world \citep{Plas2019}, and of previous experimental studies \citep{Tilman2014}. The analytical approach taken in this chapter reflects the inter-connected nature of biodiversity, ecosystem function, and external factors such as disturbance and resource availability, using Structural Equation Modelling (SEM) to 

As well as the regional study presented in \autoref{ch:befr}, I conducted two more studies which demonstrated the influence of tree biodiversity on ecosystem function in southern African woodlands. In \autoref{ch:phen}, I found that species diversity correlated with increased growing season length in drier woodlands, and earlier pre-rain green-up. In \autoref{ch:tls} I found some evidence for a species diversity effect on woodland canopy complexity. 

% What are the implications of positive biodiversity effects?
% Can protect biodiversity and increase carbon sequestration
% Miombo woodlands have carbon sequestration potential
% BUT: This thesis didn't look at the understorey, which arguably has higher biodiversity than the trees \citep{}, and we only focussed on ecossytem fcuntions that were related to biomass and carbon. We didn't look at more abstract ecosystem functions.
% In \autoref{ch:tls}, I found that tree species diversity allows miombo woodlands in particular to more effectively close their canopy and reduce light penetration to the ground. This may restrict understorey growth and reduce understorey biodiversity. Would this have a greater negative biodiversity and ecosystem function effect but in functions that we didn't measure here, such as pollination and wildlife support, possibly also water retention and limiting lateral carbon flows.
% Ecosystem service potential

% Water availability underpins positive biodiversity effects, as water availability affects tree density, species diversity, etc.

% Do we recommend inclusion of diversity information in models of the carbon cycle?
% It depends on scale. At small scales, maybe yes, but at large scales, correlations with climatic variables can do the job of species diversity (i.e. richness), while other classifications of vegetation composition might cover the job of species composition (i.e. the type of species), which can be identified from satellite imagery combinesd with expert opinion, rather than needing on the ground measurement. 
% However, this information does show us that biodiversity, and importantly the loss of biodiversity can negatively impact ecosystem function and therefore resilience. 
% We should be looking to protect biodiversity in savannas through management actions and possibly active management by planting, to preserve ecosystem function

% A key supposition in \autoref{ch:befr} was that positive biodiversity effects may be diminished due to the actions of disturbance by fire and herbivory, which reduce stem density and biomass, thus reducing the structuring effect of competition. We found that disturbance by fire reduced stem density, which in turn reduced biomass, but that disturbance actually had a positive effect on species diversity. Additionally, it was found that there was a stem density threshold below which positive biodiversity effects were diminished. 

% Miombo woodlands had stronger biodiversity effects than other vegetation types. Why? They have more resources, but are structured by disturbance by fire \citep{}, so maybe while resource limitation does impact biodiversity effects, disturbance doesn't. Is this maybe because of positive feedbacks?

% \citet{Mensah2020} found that woodlands and forests had positive tree species richness effects, while sparse savannas didn't
% \citet{Loiola2015} found that in Cerrado, fire reduced productivity through its effect on functional trait values.

% Multi-functionality, under-estimating biodiversity effects

\subsubsection{Thresholds on biodiversity effects}

In the \hyperref{ch:introduction}{introduction} to the thesis, I suggested that there may be key climatic and structural thresholds below which biodiversity effects driven by competition do not occur \citep{}. In \autoref{ch:befr}, I found that biodiversity effects become stronger with increasing organismal density, i.e. tree stem density. Finally, I found that the effect of species diversity on stand structural diversity constituted the main pathway which led to increased AGB. Together, these findings have two key implications. Firstly, even in highly disturbed ecosystems, where the role of competition is presumably reduced \citep{}, positive biodiversity effects still exist. Secondly, 


\subsection{Mechanisms for positive biodiversity effects in mesic savannas}

A multitude of previous studies have found that biodiversity positively correlates with productivity, biomass, and the resilience of these ecosystem functions to environmental perturbations, but fewer have investigated the ecological mechanisms by which biodiversity affects ecosystem function \citep{}. Identifying these pathways of influence not only improves understanding of the ecological processes underlying these observed effects, but also builds understanding of how ecosystem structure and function may change as a result of climate change or changes to disturbance regime \citep{}.

\subsubsection{Land-surface phenology}

In \autoref{ch:phen}, I found that species diversity led to increased growing season length, and earlier pre-rain green-up. 

\subsubsection{Canopy complexity}

In \autoref{ch:tls}, I found that tree species diversity in miombo woodlands correlated with increased canopy closure, taller canopies, and 

% In \autoref{ch:befr} I found that structural diversity influenced biomass, but in \autoref{ch:tls} I found no evidence that structural diversity influenced canopy complexity. 

\subsection{Regional variation in woodland structure and tree diversity across the miombo ecoregion}

Biodiversity is defined most frequently by studies of the biodiversity-ecosystem function relationship in terms of the variety of organisms which co-inahbit a given ecosystem, e.g. a woodland plot. Variation in species composition over space constitutes another perspective on biodiversity that was investigated in this study. All four investigative chapters presented in this thesis included a `vegetation type' element in their statistical analyses in order to describe variation in ecosystem function among plots that was not due to variation in the richness or breadth of organisms found within plots, but due to functional variation among plots due to the particular identity of the species found therein, and their combination. Vegetation types are often easily distinguishable, with stark differences in structure and function driven by functional differences between their dominant tree species \citep{Solbrig1996}. Distinct vegetation types arise as a result of climatic variation, disturbance history, and biogeography \citep{}. I found that southern African woodland vegetation types differed in their aboveground biomass (\autoref{ch:befr}), growing season phenological patterns (\autoref{ch:phen}), and markedly in their canopy structure (\autoref{ch:tls}). They also differed in the strength of observed biodiversity effects. 

Predictions of the effects of vegetation type on ecosystem function are necessarily conflated with the environment, which affects both species composition and ecosystem function directly (\autoref{background:plas_befr}). 

In general, miombo woodlands tended to have stronger biodiversity effects than mopane, ex-Acacia and \textit{Baikiaea} woodland types. 
% Why is that?

The plots used in the thesis cover the major savanna-woodland vegetation types in southern Africa \citep{}, notably miombo, \textit{Baikiaea}, mopane, and ex-Acacia woodlands.

We didn't cover the arid savannas, only those where fire was a dominant disturbance effect. 
% What did I find that would lead me to think that arid savannas might act different?

\section{Future directions for research}
% What did we not do and how could we improve
% Having productivity data would have made this thesis a lot better. Above-ground biomass ony gives you half the story
% Perspectives. How does this work fit into future research
% Next steps
% Environmental heterogeneity and spatial patterns at local scales to define neighbourhood interactions. GOING SMALL
% Working data on biodiversity and functional variation in vegetation into terrestrial biosphere models. Using satellite data to detect variation in biodiversity based on variation in observable function.
% Direct measurement of environmental conditions
	% Sometimes a large sample size doesn't cut it
% Reverse causation, does environment increase species diversity and ecosystem function simultaneously?
% Should diversity data be included in TBMs? Better than functional types?
	% Largest differences were among vegetation types rather than within?
% Structural data. From LiDAR, terrestrial or satellite or airborne. I found that species diversity correlates with structural diversity.


\section{Plot networks}
% And of this thesis?

\subsection{Limitations of plot-based studies}

\section{Concluding remarks}

In this thesis I demonstrated the role of tree species diversity as a determinant of ecosystem structure and function in southern African woodlands. This thesis has advanced our understanding of the vegetation dynamics of this understudied biome ...

This work raised a number of 

This highlighted

I think ...

\newpage{}
\begingroup
\setstretch{1.0}
\printbibliography[heading=subbibintoc]
\endgroup

\end{refsection}

