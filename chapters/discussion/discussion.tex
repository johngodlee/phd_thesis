\begin{refsection}

\def\chaptertitle{Discussion}


\chapter{\chaptertitle}
\label{ch:discussion}

% Intro
This thesis aimed to improve understanding of the role of tree biodiversity in shaping the structure and function of southern African woodlands. Dry tropical savanna-woodland mosaics form the dominant vegetation type in southern Africa \citep{}, though they represent the largest uncertainty in models of the terrestrial carbon cycle \citep{Ahlstrom2015}. Their vegetation dynamics are complex \citep{}, and their ecology is generally understudied compared to other dominant tropical vegetation formations \citep{}. I contextualised the studies conducted in this thesis in relation to the ``biodiversity-ecosystem function relationship'' (BEFR), which predicts positive effects of biodiversity on ecosystem function, and which is supported by hundreds of previous studies in other biomes \citep{Liang2016}. BEFR theory predicts positive biodiversity effects mainly via the mechanism of niche complementarity \citep{}, but it was unclear whether positive biodiversity effects would be observable in highly disturbed and resource-limited savanna-woodlands ecosystems \citep{}. The findings of this thesis contribute both to understanding of the determinants of ecosystem function in southern African woodlands, and to BEFR theory more generally, by testing theory formulated in temperate and wet tropical ecosystems \citep{}, in a highly disturbed and resource-limited system.

To recap, the investigative chapters of this thesis explored: 

\begin{enumerate}
	\item{The effect of tree species diversity on above-ground woody biomass across southern African mesic savannas, and mediation of this effect by environment and vegetation composition (\autoref{ch:befr}).}
	\item{Land-surface phenology as a mechanism by which tree species diversity could increase gross primary productivity (\autoref{ch:phen}).}
	\item{The role of tree species diversity and composition in driving canopy complexity (\autoref{ch:tls}).}
	\item{Regional variation in species composition and woodland structure in miombo woodlands across southern Africa, with particular reference to the understudied Hu\'{i}la plateau, Angola (\autoref{ch:bicuar}).}
\end{enumerate}

These chapters aimed to advance understanding of three broader research questions:

\begin{enumerate}
\item{Is there a detectable relationship between biodiversity and ecosystem function across southern African woodlands, and to what extent is this mediated by environment and vegetation composition?}
\item{What are the possible mechanisms driving observed biodiversity-ecosystem function relationships in southern African woodlands?}
\item{How does the tree species diversity, composition and structure of mesic savannas vary across southern Africa?}
\end{enumerate}

Here, I synthesise the key findings of the thesis, set them in the context of the principal research questions, and discuss the implications of this thesis for understanding savanna ecology and biodiversity-ecosystem function research. Furthermore, I provide a perspective for future research based on the results in this thesis.

\section{Summary of findings}
% Thesis overview
% What investigations did I conduct?
% Link back to original questions. Overall objectives
% Each chapter

\subsection{Evidence of positive biodiversity effects in southern African woodlands}

In \autoref{ch:befr}, I conducted a regional study of the effects of tree species diversity, environment, and disturbance on above-ground woody biomass (AGB) and stand structure, using a distributed network of plots from the SEOSAW network. I found that diverse plots generally held greater AGB, supporting the results of hundreds of previous studies of the biodiversity-ecosystem function relationship conducted in other natural ecosystems across the world \citep{Plas2019}, and of previous experimental studies \citep{Tilman2014}. This study however, joins only a handful of studies conducted in drylands and savannas \citep{Maestre2012, Grace2016, Plas2019, Clarke2017}. The analytical approach taken in this chapter reflects the inter-connected nature of biodiversity, ecosystem function, and external factors such as disturbance and resource availability, using Structural Equation Modelling (SEM) to effectively account for the simultaneous effect of environment on both species diversity and ecosystem function (\autoref{background:plas_befr}). This analytical framework, coupled with the regional scale dataset, raises various points about the limitations and moderating factors of biodiversity effects in this biome.

As well as the regional study presented in \autoref{ch:befr}, I conducted two more studies which demonstrated the influence of tree biodiversity on ecosystem function in southern African woodlands. In \autoref{ch:phen}, I found that species diversity correlated with increased growing season length and earlier pre-rain green-up. In \autoref{ch:tls} I found some evidence for a positive species diversity effect on woodland canopy complexity at larger spatial scales.

% What are the implications of positive biodiversity effects?
	% Can protect biodiversity and increase carbon sequestration
	% Miombo woodlands have carbon sequestration potential
	% Ecosystem service potential

% Do we recommend inclusion of diversity information in models of the carbon cycle?
% It depends on scale. At small scales, maybe yes, but at large scales, correlations with climatic variables can do the job of species diversity (i.e. richness), while other classifications of vegetation composition might cover the job of species composition (i.e. the type of species), which can be identified from satellite imagery combinesd with expert opinion, rather than needing on the ground measurement. 
% However, this information does show us that biodiversity, and importantly the loss of biodiversity can negatively impact ecosystem function and therefore resilience. 
% We should be looking to protect biodiversity in savannas through management actions and possibly active management by planting, to preserve ecosystem function

Across \autoref{ch:befr}, \autoref{ch:phen}, and \autoref{ch:tls}, water availability re-occurred as a key underpinning factor which influenced both tree species diversity and ecosystem function. 


Most studies of biodiversity effects use the variation of diversity as their measure of biodiversity, i.e. species richness. In this thesis, it was recognised that abundance evenness is another important facet of biodiversity that can provide further understanding of the mechanisms driving observed biodiversity effects. In chapter \autoref{ch:phen}, it was found that in miombo woodlands, while species richness caused an increase in cumulative EVI, longer season length, and earlier pre-rain green-up, abundance evenness caused a decrease in these metrics. This contrasting result suggests that selection effects may be driving much of the diversity effect in these woodlands, as an increase in evenness is necessarily associated with a decrease in the dominance of large canopy forming tree species which drive the phenological signal. Thus, while an increase in the number of species increases the likelihood of including these dominant species within the plot, an increase in evenness is associated with a reduction in their dominance. The dominant miombo species, mostly from the Fabaceae family, subfamily Detarioideae, grow to large canopy forming trees, with roots that can access deep groundwater reserves outside of the rainy season. Indeed, in the only non-miombo vegetation cluster encountered in data collection for \autoref{ch:phen}, abundance evenness did cause an increase in cumulative EVI and season length.

Alternatively, this result may say more about the effects of disturbance and succession on ecosystem function than they do about biodiversity effects \textit{per se}. As succession continues following a large disturbance, i.e. an intense fire, abundance evenness is expected to decrease, potentially approaching something similar to a Fisher's alpha species abundance curve, with a long tail of rare species. High abundance evenness isn't necessarily the `natural' or desirable state in wooded communities. Thus, high abundance evenness is also possibly associated with tree size and other attributes related to succession \citep{}, explaining the negative correlation of evenness with green-up lag, and cumulative EVI, as more mature woodland communities are expected to optimise productivity and resilience \citep{}.
% What do other studies say about this?

\subsubsection{Thresholds on biodiversity effects}

In the \hyperref[ch:intro]{introduction} and \hyperref[ch:background]{background} to the thesis, I suggested that there may be key climatic and structural thresholds below which biodiversity effects driven by competition do not occur in dry tropical woodlands \citep{}. This hypothesis is drawn from fundamental biodiversity-ecosystem function theory, which suggests that niche complementarity effects are predicated on inter-specific competition \citep{}. In dry tropical woodlands however, resource limitation, environmental stress, and disturbance may limit the role of competition, as community abundance is kept below carrying capacity \citep{}. In \autoref{ch:befr}, I found that biodiversity effects became stronger with increasing organismal density, i.e. tree stem density. Tree stem density was in turn driven by water availability and reduced by fire frequency. This finding suggests that there are indeed environmental factors which limit or mask the role of biodiversity in driving ecosystem function in stressful conditions. The majority of previous biodiversity-ecosystem function relationship studies have not incorporated the effects of environment, and instead tend to focus on a particular 

A number of studies in southern African woodlands have investigated ecological thresholds that determine the transition from open savanna to a closed canopy forest-like condition \citep{}. These studies have identified disturbance thresholds that maintain ``alternative stable states'' under similar climatic conditions \citep{}. My findings in \autoref{ch:befr} and \autoref{ch:tls}, suggest that tree species biodiversity itself could moderate these threshold conditions, by allowing greater biomass, greater diversity in tree physiognomy, greater canopy closure, and taller canopies, in more diverse woodlands. 

% environment. When environmental covariates were included in the model, the explanatory power of species and structural diversity decreased, but the predictive power of the model increased.

% In drier woodlands in Zambia, a strong positive biodiversity effect was observed on cumulative EVI and growing season length in vegetation where there was more moisture limitation. In higher precipitation woodlands, it appears that the dominant archetypal miombo tree species can grow to large canopy forming trees, and these individuals define cumulative EVI as a result. Meanwhile, in drier woodlands, a genuine niche complementarity effect occurs. In these drier woodlands, higher species diversity provides ecosystem level resilience to drought by increasing the breadth of water use strategies. This aligns with \citet{Ratcliffe2017}, which conducted a study in dry mediterranean woodlands 


\subsection{Mechanisms for positive biodiversity effects in mesic savannas}

A multitude of previous studies have found that biodiversity positively correlates with productivity \citep{}, AGB \citep{}, and the resilience of these ecosystem functions to environmental perturbations \citep{}, but fewer have investigated the ecological mechanisms by which biodiversity affects ecosystem function \citep{}. Identifying these pathways of effect not only improves understanding of the ecological processes underlying these observed effects, but also builds a more nuanced understanding of how ecosystem structure and function may change as a result of climate change or changes to disturbance regime \citep{}.

\subsubsection{Land-surface phenology}

In \autoref{ch:phen}, I found that species diversity led to increased growing season length, and earlier pre-rain green-up. Land-surface phenology plays and important regulating role in the global carbon, water, and nitrogen cycles \citep{Richardson2013}, and is frequently incorporated into Terrestrial Biosphere Models (TBMs), as a proxy for gross primary productivity \citep{Bloom2016}. In southern African woodlands particularly, a large body of scholarship has grown to investigate the pre-rain green-up phenomenon. My findings suggest

% variation in phenological strategy among tree species is one mechanism by which species diversity increases resilience to drought and maximises productivitty in water-limited ecosystems \citep{Stan2019, Morellato2016}. 

While pre-rain green-up has been explained by a multitude of climatic and now biodiversity variables, controls on senescence have not received as much attention \citep{}. Previous studies have suggested that senescence is largely driven by date of green-up, suggesting that resource limitation limits the length of the growing season \citep{Ryan}. In \autoref{ch:phen}, I found that the proportion of larger trees in plots also caused delayed senescence, extending the growing season and ultimately increasing productivity. This is a novel result which demonstrates the effect of large trees and older woodland patches in providing a buffer to environmental factors, i.e. the decline of water availability towards the end of the growing season. This result suggests that in deciduous water-limited savannas, large trees could also help to provide resilience to longer-term drought \citep{}, which is expected to become more prevalent as human-induced climate change progresses \citep{}.

The incorporation of biodiversity and biotic change into carbon cycle modelling has been limited \citep{Ahlstrom2015, Bodegom2011}, owing to large uncertainties in the effects of diversity on Gross Primary Productivity \citep{}, and due to difficulties measuring diversity over regional spatial scales. Our study provides a link by demonstrating a strong positive relationship between species diversity and cumulative EVI, which itself correlates with GPP \citep{Sjostrom2011}.


\subsubsection{Canopy complexity}

In \autoref{ch:tls}, I found that tree species diversity in miombo woodlands correlated with increased canopy closure, taller canopies, and increased foliage density. Effects were weaker than expected, as stochasticity in canopy structure caused by small spatial scale environmental heterogeneity, and landscape history, masked biodiversity effects.

% Increased canopy cover and foliage density, but decreased spatial heterogeneity. The decrease in spatial heterogeneity is caused by the reduction in empty space, which drives heterogeneity in sparse canopies.

Unlike in \autoref{ch:befr}, where tree size diversity was identified as an ecological mechanism by which species diversity caused increases in woody biomass, in \autoref{ch:tls} tree size variation did not have a large effect on canopy complexity. Tree size variation only affected subplot canopy layer diversity. However, species diversity did still influence both tree size variation and canopy complexity separately. This suggests that there are species-specific differences in the plasticity and physiological limits of crown shape that are independent of tree size
% Why are ch:befr and ch:tls different in this regard?

Effects of species diversity on canopy closure were stronger than the effect on foliage density.
% Why?
% So what?

% Results on the suitability of different canopy complexity metrics in sparse canopies. Foliage uniformity was poorly predicted in models. Is this because empty space dominates the canopy, while in denser forests it is dominated by foliage?

% Positive effect of Shannon on plot canopy height, foliage density. 
% Tree density not so important in plot level models.

\subsubsection{Structural diversity}

% In \autoref{ch:befr}, I found that structural diversity, i.e. variance in stem diameter and tree height, led to greater AGB, and provided an indirect pathway for the effect of species diversity on AGB.

% In \autoref{ch:befr} I found that structural diversity influenced biomass, but in \autoref{ch:tls} I found no evidence that structural diversity influenced canopy complexity. 

% Variation in functional form, links to canop complexity

% Also talk about spatial tree patterns. At small spatial scales, the spatial patterning of trees may also influence ecosystem function. In \autoref{ch:tls}, I showed that 
% Spatial mingling had opposite effects to Shannon, why is that? Particularly, a negative effect on canopy height. Did mingling correlate with some other variable? yes, it correlates with disturbance probably. mingling reduced by fire,

\subsection{Regional variation in woodland structure and tree diversity across the miombo ecoregion}

Biodiversity is defined most frequently by studies of the biodiversity-ecosystem function relationship in terms of the variety of organisms which co-inahbit a given ecosystem, e.g. a woodland plot. Variation in species composition over space constitutes another perspective on biodiversity that was investigated in this study. All four investigative chapters presented in this thesis included a `vegetation type' element in their statistical analyses in order to describe variation in ecosystem function among plots that was not due to variation in the richness or breadth of organisms found within plots, but due to functional variation among plots due to the particular identity of the species found therein, and their combination. Vegetation types are often easily distinguishable, with stark differences in structure and function driven by functional differences between their dominant tree species \citep{Solbrig1996}. Distinct vegetation types arise as a result of climatic variation, disturbance history, and biogeography \citep{}. I found that southern African woodland vegetation types differed in their aboveground biomass (\autoref{ch:befr}), growing season phenological patterns (\autoref{ch:phen}), and markedly in their canopy structure (\autoref{ch:tls}). They also differed in the strength of observed biodiversity effects. 

Predictions of the effects of vegetation type on ecosystem function are necessarily conflated with the environment, which affects both species composition and ecosystem function directly (\autoref{background:plas_befr}). 

In general, miombo woodlands tended to have stronger biodiversity effects than mopane, ex-Acacia and \textit{Baikiaea} woodland types. Miombo woodlands were found to contain more tree species than other vegetation types, and generally existed in less environmentally stressful areas. 
% \citet{Mensah2020} found that woodlands and forests had positive tree species richness effects, while sparse savannas didn't
% Miombo woodlands had stronger biodiversity effects than other vegetation types. Why? They have more resources, but are structured by disturbance by fire \citep{}, so maybe while resource limitation does impact biodiversity effects, disturbance doesn't. Is this maybe because of positive feedbacks?
% Why is that?
% Miombo woodlands differ markedly from other vegetation types found in the region in their canopy structure, stand structure, species composition, and ecosystem processes. 
% Detarioideae species

The plots used in the thesis include the major savanna-woodland vegetation types in southern Africa \citep{}, notably multiple forms of miombo, \textit{Baikiaea} woodlands, mopane woodlands, and ex-Acacia woodlands. However, arid savannas, found at the southern and north-eastern limits of the miombo ecoregion, were not included in this study. Given that in \autoref{ch:befr} it was demonstrated that water availability remained a key driving force behind the detection of biodiversity effects, and that in \autoref{ch:tls} the drier ex-Acacia savannas did not show positive biodiversity effects on canopy complexity, it is suggested that in severely water-limited savannas, biodiversity effects driven by niche complementarity may be negligible. However, facilitation effects may become more important in these systems than were detected in this thesis. \citet{Ratcliffe2017}, working in dry Mediterranean forests, found that the strength of the effect of tree species richness on many ecosystem functions increased as water availability decreased. 

In \autoref{ch:tls}, I discussed how ex-Acacia species develop sparser canopies, with shorter stature and wider crowns. This condition reduces canopy complexity and prevents the formation of a vertically stratified complex canopy. This means that acacia savannas may be less likely to close their canopy than miombo woodlands, for instance. This could be conflated with environment however, as Acacia savannas occur most often in water-limited and heavily grazed habitats, where complex canopies arwe less likely to form anyway.
% Baikiaea plots have comparable canopy closure, but lower plot foliage density than miombo plots, but this could have been a sampling issue, as we only had three Baikiaea plots

% Woodlands at the western edge of the miombo are floristically distinct from those in the east, but core `miombo' species are generally shared by sites across the region, the variation is in the smaller understorey species and those that inhabit disturbed areas. 
% In ch:tls and ch:bicuar, I showed how even with similar species composition, woodland structure differs among miombo woodlands in different parts of the miombo ecoregion. Woodlands in Tanzania had taller canopies and wider spreading crowns than those in Angola. Tanzania woodlands are warmer and wetter. Also more diverse. 
% Maybe these woodlands exist at the edge of their bioclimatic envelope and are therefore more at risk from the effects of climate change, which are predicted to reduce precipitation across the region \citep{}.


\section{Future directions for research}
% What did we not do and how could we improve
% Having productivity data would have made this thesis a lot better. AGB ony gives you half the story
% Perspectives. How does this work fit into future research
% Next steps
% Environmental heterogeneity and spatial patterns at local scales to define neighbourhood interactions. GOING SMALL
% Working data on biodiversity and functional variation in vegetation into terrestrial biosphere models. Using satellite data to detect variation in biodiversity based on variation in observable function.
% Direct measurement of environmental conditions
	% Sometimes a large sample size doesn't cut it
% Reverse causation, does environment increase species diversity and ecosystem function simultaneously?
% Should diversity data be included in TBMs? Better than functional types?
	% Largest differences were among vegetation types rather than within?
% Structural data. From LiDAR, terrestrial or satellite or airborne. I found that species diversity correlates with structural diversity.
% Other traits, move away from species, can link more easily to mechanism. Can better predict how difference vegetation types will respond to climate change and CO2 enrichment. Root:Shoot ratios. Roots contain a large amount of biomass in savannas, and has a longer residence time than aboveground wood which is more readily lost to fire and mortality.

% It was difficult for us to reliably cluster the SEOSAW plots
 
% Our study shows that biodiversity change through extensive human-induced land use change in this region will have the greatest negative impact on ecosystem function in areas of high stems density, and in certain vegetation types, specifically Mopane and ex-Acacia woodlands. T

% This thesis did not focus on the policy implications of the results. The influence of previous BEFR research on policy and management of real-world ecosystems has been limited \citep{Manning2020}. 

This thesis focussed on the biodiversity of trees, and on ecosystem functions related to trees. Southern African woodlands however, are characterised by the coexistence of trees and herbaceous understorey vegetation, notably C\textsubscript{4} grasses. Grasses in savannas have been estimated to account for between \% \citep{} and \% \citep{} of total gross primary productivity. While positive biodiversity effects were observed for ecosystem functions related to tree cover, productivity, and woody biomass, other important ecosystem functions not related to trees were ignored, such as grass biomass \citep{}, soil nutrient retention \citep{}, and wildlife fodder provision \citep{}. In \autoref{ch:tls}, I concluded that in miombo woodlands, greater tree species diversity could accelerate woody thickening and potentially woody encroachment, which would increase values of various ecosystem functions related to tree growth. The vast majority of studies of the biodiversity-ecosystem function relationship focus on ecosystem functions related to primary productivity, biomass, and carbon fixation, presumably because these are interpreted as globally important \citep{}, but possibly also because they are straightforward to measure \citep{}. Previous studies have suggested that when multiple functions are considered, that the positive effect of biodiversity increases \citep{Hector2007}, but few studies have considered combinations of ecosystem functions generated by different functional groups of species, i.e. trees and grasses \citep{}. While tree species diversity appears to promote ecosystem functions related to tree cover and biomass in southern African woodlands, it could decrease ecosystem functions generated by the non-tree portion of the biota. 

Would this have a greater negative biodiversity and ecosystem function effect but in functions that we didn't measure here, such as pollination and wildlife support, possibly also water retention and limiting lateral carbon flows.

As well as altering ecosystem function, by reducing grass growth and understorey plant production, greater woody biomass and tree cover could reduce overall biodiversity by excluding herbaceous species and the many animal species that rely upon them \citep{}. It is a frequent adage among savanna scientists that the majority of biodiversity in savannas lies in the herbaceous layer \citep{}. If the goal is to maintain and improve biodiversity in savannas, irrespective of functional group or trophic level, perhaps tree cover should be kept at an intermediate level \citep{}. While if the goal is to sequester carbon, perhaps tree biodiversity should be a greater concern \citep{}.



% BUT: This thesis didn't look at the understorey, which arguably has higher biodiversity than the trees \citep{}, and we only focussed on ecossytem fcuntions that were related to AGB and carbon. We didn't look at more abstract ecosystem functions.

\section{Plot networks}
% And of this thesis?

% Benefits of plots increase over time, so it's important to ensure that regular censuses occur, and that maintenancy is conducted.

% There are plans to include the SEOSAW plots as a supersite (\autoref{ch:legacy})

\subsection{Limitations of plot-based studies}

% Even with the one hectare plots used in the TLS study, they still included a lot of stochasticity in canopy complexity and stand structure, which made it difficult to unequivocally attribute effects to canopy complexity metrics. 
% Fewer large plots with in-depth measurements, rather than many smaller plots with fewer measurements. Who's to say that the sampled plots are taken from a normal distribution of multi-dimensional environmental space. Often they are NOT representative, or it's impossible to say if they're representative, so we need to have supersites.

\section{Concluding remarks}

In this thesis I demonstrated the role of tree species diversity as a determinant of ecosystem structure and function in southern African woodlands. This thesis has provided nuance on the vegetation dynamics of this understudied biomes. It has also contributed valuable case studies to the already broad corpus of biodiversity-ecosystem function research, testing theory developed in temperate and wet tropical systems in a highly disturbed and resource-limited system. In doing so, I have demonstrated the limits of biodiversity effects, and raised questions about the generalisability of the frequently reported biodiversity-ecosystem function relationship. I suggest that future work continues to focus on ecological mechanisms driving observed biodiversity effects, and broadens the scope of studies to include multiple functional groups of organisms along with environmental effects, to account for their inter-connected nature. I hope that future studies can leverage the current explosion in plot-based ecological monitoring data and novel data collection techniques employing state of the art technology to delve deeper into this pertinent field of study, which sits at the heart of community ecology, conservation, and earth system modelling.

\newpage{}
\begingroup
\setstretch{1.0}
\printbibliography[heading=subbibintoc]
\endgroup

\end{refsection}

