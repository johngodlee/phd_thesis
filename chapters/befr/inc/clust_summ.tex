
% Table created by stargazer v.5.2.2 by Marek Hlavac, Harvard University. E-mail: hlavac at fas.harvard.edu
% Date and time: Wed, Jul 15, 2020 - 09:38:19
\begin{table}[!htbp] \centering 
  	\caption{Description of the biogeographical clusters to which each plot in the study was assigned. Indicator species were generated using Dufrene-Legendre indicator species analysis \citep{Dufrene1997} implemented with \texttt{indval()} from the \texttt{labdsv} R package \citep{labdsv} and represent species which define the given cluster. Dominant species were identified by choosing the species with the largest mean plot level proportional AGB (Above-Ground woody Biomass) within each cluster. N = number of plots in cluster. Numeric values of species richness, stems ha\textsuperscript{-1} and AGB represent medians and interquartile ranges (75th percentile - 25th percentile).} 
  \label{clust_summ} 
\begin{tabular}{@{\extracolsep{0pt}} ccccccc} 
\\[-1.8ex]\hline 
\hline \\[-1.8ex] 
{Cluster} & {Dominant species} & {Indicator species} & {N} & \multicolumn{1}{p{2cm}}{\centering Species\\Richness} & \multicolumn{1}{p{2.5cm}}{\centering Stem density\\(stems ha\textsuperscript{-1})} & \multicolumn{1}{p{2cm}}{\centering AGB\\(t ha\textsuperscript{-1})} \\
\hline \\[-1.8ex] 
Core miombo & \begin{tabular}[c]{@{}c@{}c@{}} \textit{Brachystegia spiciformis} \\\textit{Julbernardia paniculata} \\\textit{Brachystegia boehmii} \end{tabular} & \begin{tabular}[c]{@{}c@{}c@{}} \textit{Parinari curatellifolia} \\\textit{Uapaca kirkiana} \\\textit{Brachystegia spiciformis} \end{tabular} & 523 & 20(16.9) & 204(142.5) & 44.2(36.11) \\ 
\hline
ex-Acacia & \begin{tabular}[c]{@{}c@{}c@{}} \textit{Spirostachys africana} \\\textit{Senegalia burkei} \\\textit{Senegalia nigrescens} \end{tabular} & \begin{tabular}[c]{@{}c@{}c@{}} \textit{Euclea racemosa} \\\textit{Vachellia nilotica} \\\textit{Spirostachys africana} \end{tabular} & 188 & 12(10.3) & 181(166.5) & 54.5(61.33) \\ 
\hline
Mopane & \begin{tabular}[c]{@{}c@{}c@{}} \textit{Colophospermum mopane} \\\textit{Androstachys johnsonii} \\\textit{Kirkia acuminata} \end{tabular} & \begin{tabular}[c]{@{}c@{}c@{}} \textit{\textit{Colophospermum mopane}} \\\textit{Psuedolachnostylis maprouneifolia} \\\textit{Lannea discolor} \end{tabular} & 58 & 10(10.2) & 186(125.6) & 42.7(32.83) \\ 
\hline
Sparse miombo / Baikiaea & \begin{tabular}[c]{@{}c@{}c@{}} \textit{Baikiaea plurijuga} \\\textit{Burkea africana} \\\textit{Pterocarpus angolensis} \end{tabular} & \begin{tabular}[c]{@{}c@{}c@{}} \textit{Burkea africana} \\\textit{Baikiaea plurijuga} \\\textit{Pterocarpus angolensis} \end{tabular} & 466 & 12(13.7) & 178(129.5) & 36.9(26.98) \\ 
\hline
\hline \\[-1.8ex] 
\end{tabular} 
\end{table} 
