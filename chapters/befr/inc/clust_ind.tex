\begin{table}[h]
	\caption[Floristic description of vegetation types]{Floristic description of vegetation types, showing the top three indicator and dominant species for each vegetation type.} 
	\label{befr:clust_ind} 
\begin{tabular}{lrr} 
  \toprule
{Cluster} & {Dominant species} & {Indicator species} \\
\midrule
\multirow{3}{*}{Core miombo} & \textit{Brachystegia spiciformis} & \textit{Parinari curatellifolia} \\
 & \textit{Julbernardia paniculata} & \textit{Uapaca kirkiana} \\
 & \textit{Brachystegia boehmii} & \textit{Brachystegia spiciformis} \\  
\midrule 
\multirow{3}{*}{ex-Acacia} & \textit{Spirostachys africana} & \textit{Euclea racemosa} \\
 & \textit{Senegalia burkei} & \textit{Vachellia nilotica}\\
 & \textit{Senegalia nigrescens} & \textit{Spirostachys africana} \\
\midrule
\multirow{3}{*}{Mopane} & \textit{Colophospermum mopane} & \textit{Colophospermum mopane} \\
 & \textit{Androstachys johnsonii} & \textit{Psuedolachnostylis maprouneifolia} \\
 & \textit{Kirkia acuminata} & \textit{Lannea discolor} \\ 
\midrule
\multirow{3}{*}{Baikiaea} & \textit{Baikiaea plurijuga} & \textit{Burkea africana} \\
 & \textit{Burkea africana} & \textit{Baikiaea plurijuga} \\
 & \textit{Pterocarpus angolensis} & \textit{Pterocarpus angolensis} \\
\bottomrule
\end{tabular} 
\end{table} 
