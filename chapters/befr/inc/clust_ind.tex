\begin{table}[tb] \centering 
  	\caption[Floristic description of vegetation types]{Indicator species were generated using Dufrene-Legendre indicator species analysis \citep{Dufrene1997} implemented with \texttt{indval()} from the \texttt{labdsv} R package \citep{labdsv} and represent species which define the given cluster. Dominant species were identified by choosing the species with the largest mean plot level proportional AGB (Above-Ground woody Biomass) within each cluster.} 
  \label{befr:clust_ind} 
\begin{tabular}{ccc} 
  \hline
{Cluster} & {Dominant species} & {Indicator species} \\
\hline
Core miombo & \begin{tabular}[c]{@{}c@{}c@{}} \textit{Brachystegia spiciformis} \\\textit{Julbernardia paniculata} \\\textit{Brachystegia boehmii} \end{tabular} & \begin{tabular}[c]{@{}c@{}c@{}} \textit{Parinari curatellifolia} \\\textit{Uapaca kirkiana} \\\textit{Brachystegia spiciformis} \end{tabular} \\ 
\hline
ex-Acacia & \begin{tabular}[c]{@{}c@{}c@{}} \textit{Spirostachys africana} \\\textit{Senegalia burkei} \\\textit{Senegalia nigrescens} \end{tabular} & \begin{tabular}[c]{@{}c@{}c@{}} \textit{Euclea racemosa} \\\textit{Vachellia nilotica} \\\textit{Spirostachys africana} \end{tabular} \\ 
\hline
Mopane & \begin{tabular}[c]{@{}c@{}c@{}} \textit{Colophospermum mopane} \\\textit{Androstachys johnsonii} \\\textit{Kirkia acuminata} \end{tabular} & \begin{tabular}[c]{@{}c@{}c@{}} \textit{\textit{Colophospermum mopane}} \\\textit{Psuedolachnostylis maprouneifolia} \\\textit{Lannea discolor} \end{tabular} \\ 
\hline
\multicolumn{1}{p{3.5cm}}{\centering Sparse miombo/\\Baikiaea} & \begin{tabular}[c]{@{}c@{}c@{}} \textit{Baikiaea plurijuga} \\\textit{Burkea africana} \\\textit{Pterocarpus angolensis} \end{tabular} & \begin{tabular}[c]{@{}c@{}c@{}} \textit{Burkea africana} \\\textit{Baikiaea plurijuga} \\\textit{Pterocarpus angolensis} \end{tabular} \\ 
\hline
\end{tabular} 
\end{table} 
