\chapter*{Abstract}

\label{abstract}

A broad corpus of previous research has sought to understand the role of biodiversity as a driver of ecosystem structure and function. Although theory suggests that increased biodiversity should increase ecosystem function by niche complementarity among co-existing species, in natural systems wide variation in the biodiversity effect exists among vegetation types and along environmental gradients. In southern African woodlands and savannas, which experience disturbance by fire and herbivory, drought and extreme temperatures, it is unclear whether positive biodiversity effects should occur. In this thesis, I explore the ecology of southern African woodlands through the lens of the biodiversity-ecosystem function relationship, to improve our understanding of the role of tree diversity as a mediator of ecosystem function, its interactions with abiotic environment, and its effect on woodland structure. 

In temperate and wet tropical forests, where the majority of biodiversity-ecosystem function studies in natural woody vegetation have been conducted, the positive effect of niche complementarity hinges on the condition that conspecific competition is the limiting factor to ecosystem function. In highly disturbed and environmentally stressed systems however, this may not hold true. I conducted a regional study investigating the role of tree species diversity and structural diversity as mediators of woody biomass, using a plot network of 1235 plots spanning wide climatic and biogeographic gradients across southern Africa. Using Structural Equation Modelling, I determined that tree species diversity has a positive effect on biomass, operating mostly via its effect on structural diversity. I found that biodiversity itself increases with water availability, and that positive biodiversity effects only arise under sufficiently high stem density.

To further understand the ecological mechanisms which drive positive biodiversity-productivity relationships, I explored the effects of tree species diversity and woodland demographic structure on patterns of land-surface phenology. I combined a dense plot-based tree census dataset across multiple deciduous Zambian woodland types with remotely sensed measures of green-ness, to understand drivers of variation in pre-rain green-up, growing season length and productivity. I found that pre-rain green-up occurred earlier in more diverse sites, across all woodland types, while in non-miombo woodlands, species richness also increased post-rain senescence lag and season length. I also found that large-sized trees increase the degree of both pre-rain green-up and post-rain senescence lag, across vegetation types, with an effect size similar to that of species richness.

Southern African woodlands occur as a complex mosaic of open grassy patches and closed canopy forest-like patches, driven by positive feedbacks of fire-induced tree mortality and grass growth, but the biotic mechanisms causing variation in canopy closure are unclear. I used terrestrial LiDAR at two sites, in Tanzania and Angola, to understand at fine spatial scale the effects of species composition and diversity on canopy architecture and canopy cover. Species diversity was found to allow increased spatial clumping of trees, which drove vertical canopy layer diversity and canopy height, demonstrating an indirect role of species diversity on canopy cover via stand structure. Taken together with the regional study of the biodiversity-ecosystem function relationship, these findings suggest a nuanced role of tree species diversity on ecosystem function, operating primarily via its effect on canopy structural diversity in southern African woodlands. I propose that higher diversity communities are more likely to produce forest-like closed canopy woodlands, with a higher upper limit on biomass, and are more likely to transition from savanna to closed canopy forest under conditions of atmospheric CO\textsubscript{2} enrichment.

Finally, in an effort to increase our understanding of the variation in diversity and structure of woodlands across southern Africa, I conducted a study of tree species biodiversity and woodland structure in Bicuar National Park, southwest Angola, with comparison to other woodlands around the miombo ecoregion. Much of the published plot data and woodland monitoring infrastructure in miombo woodlands is located in central and eastern regions of southern Africa, while woodlands in the west of the region, which occur entirely within Angola, remain poorly represented. I found that Bicuar National Park constitutes an important woodland refuge at the transition between dry miombo woodland and \textit{Baikiaea}-\textit{Baphia} woodlands. I recorded 27 tree species not recorded elsewhere in the miombo ecoregion outside the Hu\'{i}la plateau. An additional study of one-off plots in areas previously disturbed by shifting cultivation, found that this disturbance increases tree species diversity, but ultimately reduces woody biomass, even after a period of regeneration, potentially representing a directional shift to a different stable vegetation type.

Together, the findings of this thesis demonstrate multiple relationships among tree biodiversity, ecosystem structure, and ecosystem function, measured primarily through woody biomass and productivity, at multiple spatial scales. I conclude that incorporation of diversity and canopy structural information into earth system models, by scaling up plot data using cutting edge remotely sensed datasets, could improve predictions of how climate change and biodiversity change will impact the functioning of different vegetation types across southern Africa, with consequences for carbon cycle modelling, conservation management, and ecosystem service provision. Finally, I suggest that biodiversity loss of large archetypal miombo tree species will have the greatest impact on a number of ecosystem functions related to carbon cycling, raising concerns over the impacts of selective logging of these species.

