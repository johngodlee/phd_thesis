\chapter*{Lay summary}
\label{lay}

% Dry tropical woodlands 

At the nexus of these environmental forcings are the plant communities comprising these ecosystems, which 

% Define ecosystem function
Ecosystem functions are defined , such as converting atmospheric \co to woody biomass, or providing ecosystem resilience to environmental change. 

The individual plants within an ecosystem act as a mediator interface between environment and function. It is therefore reasonable to expect that the composition, structure, and biodiversity of these ecosystems could affect the way the functioning b

% Biodiversity and ecosystem function
In other biomes, many studies have shown that as plant biodiversity increases, so does ecosystem function. These findings are supported by theory, which predicts that ecosystems with more species will be able to make more complete use of available resources, such as light, water, and space, due to differences among species in their growth strategy. 

% What does this thesis do?
In this thesis, I explore how ecosystem functions and ecosystem properties vary across southern African woodlands with variation in tree species composition, species diversity, and environmental conditions. My aim is to improve our understanding of 

The thesis is split into four main parts, each of which explores variation in diversity and it's effect on function across southern Africa.

% First chapter - BEFR
Using a large network of woodland plots, across 9 countries, I investigated how tree species diversity and tree size diversity mediated the effects of various environmental drivers of woody biomass. I found that plots with more tree species also had greater variation in tree size, and that this was the main driver of increased woody biomass, by allowing more individuals to inhabit a smaller space. I also found that positive effects of species diversity on biomass only occur when trees are sufficiently close together. I suggest that below this tree density threshold, the positive effects of biodiversity on biomass are not realised as individuals do not compete with each other.

% Second chapter - Phen
The timing of leaf production is an important driver of productivity and growth in trees. Leaves are the primary interface between the atmosphere and the plant, so without them, the plant cannot absorb atmospheric carbon needed for growth. I investigated the effects of tree species diversity on leaf production using a dense network of \textapprox{}600 woodland plots in Zambia. I found that plots with greater species diversity had a prolonged growing season, and were able to produce leaves earlier than plots with fewer species. I also found differences in the leaf production behaviour of woodlands based on their dominant species. These findings will help us to predict how fine-scale variation in biodiversity and vegetation composition will impact carbon cycling across the dry tropics.

% Third chapter - TLS

% Fourth chapter - Bicuar

% Summ up
