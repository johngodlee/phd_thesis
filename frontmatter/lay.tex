\chapter*{Lay summary}
\label{lay}

% Dry tropical woodlands, what don't we know?
The tropical woodlands and savannas of southern Africa are complex and heterogeneous ecological systems, defined by the coexistence of grasses and trees. They vary widely in species composition, tree cover and canopy structure, from the dry Acacia savannas of South Africa, to the dense and humid miombo woodlands of southern Congo. Ecosystem functions in tropical savannas depend on a combination of factors, including rainfall, soil fertility, disturbance from seasonal grass fires, and herbivore grazing. Understanding what controls the rate of tree growth (i.e. productivity) and the total woody biomass in tropical savannas is important for predicting how these ecosystems contribute to the global carbon cycle, as productivity sets the rate at which trees take carbon from the atmosphere and store it in woody biomass. Yet, despite knowing the environmental drivers of savanna ecosystem processes, we don't have a detailed understanding of how variation in tree species composition and structure mediates the effect of environment on ecosystem function. 

% Biodiversity and ecosystem function
In other biomes, many studies have shown that as plant biodiversity increases, so does ecosystem function. These findings are supported by theory, which predicts that ecosystems with more species will be able to make more complete use of available resources, such as light, water, and space, due to differences among species in their growth strategy. 

% What does this thesis do?
In this thesis, I explore how tree species composition and species diversity mediate the effects of environment on ecosystem functions and ecosystem properties across southern African woodlands. My aims are 1) to add depth to our understanding of savanna ecology, and 2) to extend general theory about the effects of biodiversity on ecosystem function to ecosystems which experience disturbance and resource-scarcity. 

% First chapter - BEFR
Using a large network of woodland plots, across nine countries, I investigated how tree species diversity and tree size diversity mediated the effects of various environmental drivers of woody biomass. I found that plots with more tree species also had greater variation in tree size, and that this was the main driver of increased woody biomass, by allowing more individuals to pack together in a smaller space, by reducing competition. I also found that positive effects of species diversity on biomass only occur when trees are sufficiently close together, which is in turn controlled by resource availability. I suggest that below this tree density threshold, the positive effects of biodiversity on biomass are not realised as individuals do not compete with each other to the degree that it limits their biomass.

% Second chapter - Phen
The timing of leaf production in response to seasonal rainfall is an important driver of productivity and growth in trees. Leaves are the primary interface between the atmosphere and the plant, without them the plant cannot absorb atmospheric carbon needed for growth. I investigated the effects of tree species diversity on leaf production using a dense network of \textasciitilde{}600 woodland plots in Zambia. I found that plots with greater species diversity had a longer growing season, and were able to produce leaves earlier in the rainy season than plots with fewer species. I also found differences in the leaf production behaviour of woodlands based on their dominant species. These findings help to understand how fine-scale variation in biodiversity and vegetation composition impact carbon cycling across the dry tropics.

% Third chapter - TLS
I used terrestrial LiDAR (laser-scanning) to measure savanna canopy structure in fine detail, to understand how variation in tree species richness and spatial patterns of tree diversity affect canopy cover and canopy structural complexity. I found that spatial clustering of individuals increased with species richness, allowing greater canopy cover and more complete use of available light in diverse savannas. This finding suggests that more diverse savannas are more likely to transition from savanna to closed canopy forest in the future.

% Fourth chapter - Bicuar
Finally, much of the long-term plot-based research in southern African woodlands comes from Tanzania, Mozambique, and Zambia, while woodlands in the western part of southern Africa remain understudied, and much less is known about their diversity and vegetation dynamics. I conducted a study of tree species composition and woodland structure using 15 permanent plots constructed in Bicuar National Park, southwest Angola, to highlight the divergent floristic composition and conservation value of this region. I encountered many species not recorded elsewhere in the region, and demonstrated the effects of pervasive agriculture on the diversity of these woodlands. 

% Summ up
Together, this thesis demonstrates multiple mechanisms by which species diversity in southern African woodlands affect ecosystem function, measured as woody biomass and productivity. I highlight the wide functional variation of savanna vegetation across the region. I conclude that incorporating information on species diversity and canopy structure into carbon cycling models could improve predictions of how climate and biodiversity change will impact the functioning of different vegetation types across southern Africa. I also suggest that biodiversity loss of keystone miombo tree species will have the greatest negative effect on ecosystem function in this biome.
